\documentclass[12pt,a4paper]{report}

\usepackage{alltt, fancyvrb, url}
\usepackage{graphicx}
\usepackage[utf8]{inputenc}
\usepackage{float}
\usepackage{xcolor}
\usepackage{hyperref}
\usepackage{longtable}
\usepackage{listings}
\usepackage{color}
\usepackage[utf8]{inputenc}

\usepackage{enumitem}
\usepackage{titlesec} % aggiunto per controllare gli spazi tra titoli

% Riduce gli spazi verticali generali
\setlength{\parskip}{0pt}

% Nuova formattazione compatta del chapter per avvicinare le sezioni
\titleformat{\chapter}[display]
  {\normalfont\huge\bfseries}
  {\chaptertitlename\ \thechapter}{10pt}{\Huge}
% {left}{before-sep}{after-sep} -> after-sep negativo per ridurre gap tra chapter e section
\titlespacing*{\chapter}{0pt}{-30pt}{6pt}

% Liste compatte: rimuove spazio tra item e spazio prima/dopo la lista
\setlist[itemize]{noitemsep, topsep=0pt, parsep=0pt, partopsep=0pt, leftmargin=*}
\setlist[enumerate]{noitemsep, topsep=0pt, parsep=0pt, partopsep=0pt}

\usepackage{amsmath}
\usepackage{geometry}
\usepackage{pdfpages}
\usepackage{caption}
\usepackage{booktabs}
\usepackage[utf8]{inputenc}

\geometry{margin=1in}

\definecolor{codegray}{rgb}{0.5,0.5,0.5}
\definecolor{codepurple}{rgb}{0.58,0,0.82}
\definecolor{backcolour}{rgb}{0.95,0.95,0.92}

\usepackage{newlfont}
\usepackage{gensymb}

\usepackage[italian]{babel}
\usepackage[capitalise, italian]{cleveref}

\usepackage{fancyhdr}
\usepackage{lastpage}

\graphicspath{{./src}}

\fancypagestyle{soloNumero}{
    \fancyhf{} % Pulisce tutto
    \renewcommand{\headrulewidth}{0pt} % Rimuove linea superiore
    \renewcommand{\footrulewidth}{0pt} % Rimuove linea inferiore
    \fancyfoot[R]{\thepage} % Numero solo a destra
}\pagestyle{fancy}
\fancyhf{}

% --- HEADER (Testata) ---
% Titolo pulito a sinistra (pari), Capitolo in Maiuscoletto a destra (dispari)
\fancyhead[LE]{AlmaMensa - Gestione Mense Universitarie}
\fancyhead[RO]{\textsc{\nouppercase{\leftmark}}} % Small Caps e rimuove il forcing del maiuscolo

% --- FOOTER (Piè di pagina) ---
\fancyfoot[LE,RO]{\thepage}
\fancyfoot[LO,RE]{\footnotesize Tecnologie Web} % Font leggermente più piccolo per il corso

% --- STILE LINEE ---
\renewcommand{\headrulewidth}{0.4pt}
\renewcommand{\footrulewidth}{0.4pt}

% --- RIDEFINIZIONE STILE PLAIN (Pagine inizio capitolo) ---
\fancypagestyle{plain}{
  \fancyhf{}
  \fancyhead[L]{\footnotesize AlmaMensa - Gestione Mense Universitarie}
  \fancyfoot[R]{\thepage}
  \renewcommand{\headrulewidth}{0.4pt}
  \renewcommand{\footrulewidth}{0pt} % Rimuove la linea inferiore nelle pagine d'inizio
}

% --- CONFIGURAZIONE MARKS ---
\renewcommand{\chaptermark}[1]{%
  \markboth{#1}{}% Memorizza solo il titolo senza "Capitolo X" per massima leggerezza
}

\textwidth=450pt\oddsidemargin=0pt
\begin{document}

\begin{titlepage}
    \begin{center}
        {{\Large{\textsc{Alma Mater Studiorum $\cdot$ Università di Bologna}}}} \rule[0.1cm]{15.8cm}{0.1mm}
        \rule[0.5cm]{15.8cm}{0.6mm}
        {\small{\bf CORSO DI LAUREA IN INGEGNERIA E SCIENZE INFORMATICHE \\ A.A. 2025/26 }}
    \end{center}
    \vspace{15mm}
    \begin{center}
        {\LARGE{\bf AlmaMensa - Gestione Mense Universitarie}}
    \end{center}
    \begin{center}
        {\LARGE Relazione per il corso di Tecnologie Web}
    \end{center}

    \vspace{8mm}
    \begin{center}
        \includegraphics[width=0.8\textwidth]{Copertina}
    \end{center}
    \vspace{10mm}

    {\large{\bf \noindent
    Componenti del gruppo:\\}
    Bartocetti Enrico, matr. 0001115097\\
    Benedetti Nicholas, matr. 0001114021\\
    Tazzieri Nicolas, matr. 0001114078}

\end{titlepage}

\tableofcontents

\chapter{Analisi}

\section{Requisiti}

Il campus di Cesena dell'Università di Bologna vuole offrire un nuovo servizio a tutti i suoi studenti e al suo personale: data la totale assenza di punti ristoro, si vuole creare un unico portale per la gestione delle mense nel territorio cesenate.
\\
Il sistema dovrà prevedere le seguenti tipologie di utenze:
\begin{itemize}
    \item Utente non registrato
    \item Cliente
    \item Gestore mensa
\end{itemize}

Ogni utente sarà comunque in grado di \textbf{modificare} i propri dati personali.

\subsection{Utente non registrato}
Un utente non registrato potrà comunque accedere al portale, ma sarà in grado solo di \textbf{visualizzare} il nominativo, la posizione, i menu e le eventuali recensioni delle mense registrate al sistema.

\subsection{Cliente}
Per cliente si intendono tutti gli utenti che potranno beneficiare del servizio mense, ovvero studenti, docenti, personale tecnico e amministravo.
\\
Questa tipologia di utente sarà in grado di visualizzare i \textbf{posti rimanenti} nelle varie mense per poterne \textbf{prenotare} uno, disdire la prenotazione e aggiungere \textbf{recensioni} per una mensa visitata. Per registrare l'\textbf{effettiva presenza}, dovrà mostrare il qr-code presente nell'area personale al suo ingresso nella mensa.

\subsection{Gestore mensa}
\label{sub:GestoreMensa}
Ogni mensa avrà delle credenziali, generate da un amministratore al momento della registrazione al sistema, che ne permeteranno l'accesso alla piattaforma.
\\
Tramite la propria area una mensa potrà pubblicare i \textbf{menu giornalieri}, visualizzare le \textbf{prenotazioni} e registrare le \textbf{presenze} dei clienti.
\chapter{Fase di progettazione}
\section{Obiettivi}
La progettazione di AlmaMensa è stata guidata dai principi dello User Centered Design, concentrandoci nell'offrire anche a utenti non affini all'informatica una buona user experience. Le scelte di design sono focalizzate su:
\begin{itemize}
\item \textbf{Accessibilità}: il design mira a soddisfare i requisiti di livello AA imposti dalle WCAG, curando sia la scelta del font e del contrasto tra i colori, sia la navigazione tramite screenreaders.
\item \textbf{Estetica}: è stato adottato un approccio minimale, basato sulla presentazione di informazioni essenziali tramite cards, che poi possono essere approfondite dall'utente in pagine dedicate. Il look\&feel è quello di una qualsiasi applicazione sviluppata col framework Bootstrap. Per riprendere l'identità visiva del'Alma Mater e trasmettere famigliarità agli utenti target (studenti del campus di Cesena), è stato adottato come colore principale quello delle pagine istituzionali dell'Università di Bologna.
\item \textbf{Mobile-first}: l'interfaccia è stata progettata partendo dai dispositivi mobile e scalata su desktop tramite Bootstrap, ottenendo come risultato un sito web responsive e fruibile su tutte le tipologie di dispositivi in maniera accettabile.
\end{itemize}
\section{Sviluppo}
Lo sviluppo dell'applicativo è stato svolto seguendo questi passaggi:
\begin{enumerate}
    \item Ricerca dell'ambito su cui sviluppare l'applicazione, chiedendo a diversi colleghi universitari potenziali servizi utili alla vita universitaria
    \item Creazione di un focus group (formato anche da persone non competenti in informatica) per la discussione delle funzionalità di un servizio di mensa, rilevato come potenziale necessità degli studenti nel campus di Cesena.
    \item Produzione dei mockup, adottando un approccio mobile-first.
    \item Presentazione e conseguente valutazione dei mockup dal focus group.
    \item Rifinitura dei mockup a seguito della valutazione.
    \item Iterazione dei passaggi 3-5, finchè non è ottenuto un design soddisfacente.
    \item Sviluppo dell'applicazione, ritorno al punto 3 nel caso di incongruenze non trovate precedentemente.
\end{enumerate}
\section{Architettura di massima}
Per la gestione della maggior parte delle pagine è stato adottato un approccio template-based lato server (PHP). Per poter gestire alcune funzionalità particolari, come la mappa interattiva, la prenotazione e la rilevazione delle presenze tramite QR code, sono stati creati degli script dedicati che comunicano con le API del server tramite richieste AJAX, consentendo l'aggiornamento dinamico dei contenuti.

\section{Database}
Nella pagina seguente si riporta lo schema ER da cui è stato estratto il database alla base dell'applicativo.

Data la presenza degli attributi ridondanti \texttt{media\_recensioni} e \texttt{num\_recensioni} nelle mense, è stata creata una stored procedure che li aggiorna sulla base delle recensioni: la procedura viene richiamata per ogni mensa coinvolta in azioni di \texttt{INSERT, UPDATE o DELETE} nella tabella delle recensioni.

Per la gestione delle associazioni ad anello su \texttt{Mensa - Menu - Piatti}, è stata creata un'altra procedura per controllare che i piatti inseriti in un menu siano proposti dalla stessa mensa, visto che tale vincolo non è modellato a livello di schema.

\includepdf{src/AlmaMensa_ER.pdf}
\end{document}
