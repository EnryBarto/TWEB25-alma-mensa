\documentclass[12pt,a4paper]{report}

\usepackage{alltt, fancyvrb, url}
\usepackage{graphicx}
\usepackage[utf8]{inputenc}
\usepackage{float}
\usepackage{xcolor}
\usepackage{hyperref}
\usepackage{longtable}
\usepackage{listings}
\usepackage{color}
\usepackage[utf8]{inputenc}

\usepackage{enumitem}
\usepackage{amsmath}
\usepackage{geometry}
\usepackage{pdfpages}
\usepackage{caption}
\usepackage{booktabs}
\usepackage[utf8]{inputenc}

\geometry{margin=1in}

\definecolor{codegray}{rgb}{0.5,0.5,0.5}
\definecolor{codepurple}{rgb}{0.58,0,0.82}
\definecolor{backcolour}{rgb}{0.95,0.95,0.92}

\usepackage{newlfont}
\usepackage{gensymb}

\usepackage[italian]{babel}
\usepackage[capitalise, italian]{cleveref}

\usepackage{fancyhdr}
\usepackage{lastpage}

\graphicspath{{./src/img}}

\fancypagestyle{soloNumero}{
    \fancyhf{} % Pulisce tutto
    \renewcommand{\headrulewidth}{0pt} % Rimuove linea superiore
    \renewcommand{\footrulewidth}{0pt} % Rimuove linea inferiore
    \fancyfoot[R]{\thepage} % Numero solo a destra
}\pagestyle{fancy}
\fancyhf{}

% --- HEADER (Testata) ---
% Titolo pulito a sinistra (pari), Capitolo in Maiuscoletto a destra (dispari)
\fancyhead[LE]{AlmaMensa - Gestione Mense Universitarie}
\fancyhead[RO]{\textsc{\nouppercase{\leftmark}}} % Small Caps e rimuove il forcing del maiuscolo

% --- FOOTER (Piè di pagina) ---
\fancyfoot[LE,RO]{\thepage}
\fancyfoot[LO,RE]{\footnotesize Tecnologie Web} % Font leggermente più piccolo per il corso

% --- STILE LINEE ---
\renewcommand{\headrulewidth}{0.4pt}
\renewcommand{\footrulewidth}{0.4pt}

% --- RIDEFINIZIONE STILE PLAIN (Pagine inizio capitolo) ---
\fancypagestyle{plain}{
  \fancyhf{}
  \fancyhead[L]{\footnotesize AlmaMensa - Gestione Mense Universitarie}
  \fancyfoot[R]{\thepage}
  \renewcommand{\headrulewidth}{0.4pt}
  \renewcommand{\footrulewidth}{0pt} % Rimuove la linea inferiore nelle pagine d'inizio
}

% --- CONFIGURAZIONE MARKS ---
\renewcommand{\chaptermark}[1]{%
  \markboth{#1}{}% Memorizza solo il titolo senza "Capitolo X" per massima leggerezza
}

\textwidth=450pt\oddsidemargin=0pt
\begin{document}

\begin{titlepage}
    \begin{center}
        {{\Large{\textsc{Alma Mater Studiorum $\cdot$ Università di Bologna}}}} \rule[0.1cm]{15.8cm}{0.1mm}
        \rule[0.5cm]{15.8cm}{0.6mm}
        {\small{\bf CORSO DI LAUREA IN INGEGNERIA E SCIENZE INFORMATICHE \\ A.A. 2025/26 }}
    \end{center}
    \vspace{15mm}
    \begin{center}
        {\LARGE{\bf AlmaMensa - Gestione Mense Universitarie}}
    \end{center}
    \begin{center}
        {\LARGE Relazione per il corso di Tecnologie Web}
    \end{center}

    \vspace{8mm}
    \begin{center}
        \includegraphics[width=0.8\textwidth]{Copertina}
    \end{center}
    \vspace{10mm}

    {\large{\bf \noindent
    Componenti del gruppo:\\}
    Bartocetti Enrico, matr. 0001115097\\
    Benedetti Nicholas, matr. 0001114021\\
    Tazzieri Nicolas, matr. 0001114078}

\end{titlepage}

\tableofcontents

\chapter{Analisi}

\section{Requisiti}

Il campus di Cesena dell'Università di Bologna vuole offrire un nuovo servizio a tutti i suoi studenti e al suo personale: data la totale assenza di punti ristoro, si vuole creare un unico portale per la gestione delle mense nel territorio cesenate.
\\
Il sistema dovrà prevedere le seguenti tipologie di utenze:
\begin{itemize}
    \item Utente non registrato
    \item Cliente
    \item Gestore mensa
\end{itemize}

Ogni utente sarà comunque in grado di \textbf{modificare} i propri dati personali.

\subsection{Utente non registrato}
Un utente non registrato potrà comunque accedere al portale, ma sarà in grado solo di \textbf{visualizzare} il nominativo, la posizione, i menu e le eventuali recensioni delle mense registrate al sistema.

\subsection{Cliente}
Per cliente si intendono tutti gli utenti che potranno beneficiare del servizio mense, ovvero studenti, docenti, personale tecnico e amministravo.
\\
Questa tipologia di utente sarà in grado di visualizzare i \textbf{posti rimanenti} nelle varie mense per poterne \textbf{prenotare} uno, disdire la prenotazione e aggiungere \textbf{recensioni} per una mensa visitata. Per registrare l'\textbf{effettiva presenza}, dovrà mostrare il qr-code presente nell'area personale al suo ingresso nella mensa.

\subsection{Gestore mensa}
\label{sub:GestoreMensa}
Ogni mensa avrà delle credenziali, generate da un amministratore al momento della registrazione al sistema, che ne permeteranno l'accesso alla piattaforma.
\\
Tramite la propria area una mensa potrà pubblicare i \textbf{menu giornalieri}, visualizzare le \textbf{prenotazioni} e registrare le \textbf{presenze} dei clienti.

\end{document}
